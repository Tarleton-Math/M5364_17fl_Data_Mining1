\documentclass[letterpaper]{article}
%Generally useful packages
\usepackage{amsmath}
\usepackage{amsthm}
\usepackage{amssymb}
\usepackage{verbatim}
\usepackage{epsfig}
\usepackage{float} 
\usepackage{graphicx}
\usepackage{epstopdf}
\usepackage{units}
\usepackage{pdfpages}
\usepackage{comment}
\usepackage{xparse}
\usepackage{xspace}
\usepackage{hyperref}
\hypersetup{
	colorlinks=true,
	linkcolor=cyan,
	filecolor=magenta,      
	urlcolor=blue,
}
\usepackage[english]{babel}
\usepackage[autostyle, english = american]{csquotes}
\MakeOuterQuote{"}


%Convenince macros for common tasks

%Typesetting
\usepackage[shortlabels]{enumitem}
\setlist{itemsep=0ex,parsep=\parskip,topsep=0ex,partopsep=0ex}
\newlist{Mitem}{itemize}{5}
\setlist[Mitem,1]{label=$\bullet$}
\setlist[Mitem,2]{label=$-$}
\setlist[Mitem,3]{label=$\ast$}
\setlist[Mitem,4]{label=$\circ$}
\setlist[Mitem,5]{label=$\cdot$}
%\DeclareDocumentCommand\Mit{ O{} O{} }{\begin{Mitem}[#1] #2 \end{Mitem}}
\DeclareDocumentCommand\bit{ O{} }{\begin{Mitem}[#1]}
\DeclareDocumentCommand\eit{}{\end{Mitem}}

\newlist{Menum}{enumerate}{5}
\setlist[Menum,1]{label=\arabic*)}
\setlist[Menum,2]{label=\arabic{Menumi}.\arabic*)}
\setlist[Menum,3]{label=\arabic{Menumi}.\Alph{Menumii}.\arabic*)}
\setlist[Menum,4]{label=\arabic{Menumi}.\Alph{Menumii}.\roman{Menumiii}.\arabic*)}
\setlist[Menum,5]{label=\arabic{Menumi}.\Alph{Menumii}.\roman{Menumiii}.\alph{Menumiii}.\arabic*)}
%\DeclareDocumentCommand\Men{ O{} O{} }{\begin{Menum}[#1] #2 \end{Menum}}
\DeclareDocumentCommand\ben{ O{} }{\begin{Menum}[#1]}
\DeclareDocumentCommand\een{}{\end{Menum}}


\DeclareDocumentCommand\hhrule{}{\hrule \vspace{1\lineskip} \hrule}
\DeclareDocumentCommand\bu{ O{} }{\textbf{\underline{#1}}}
\DeclareDocumentCommand\seeme{ O{} }{\textbf{\underline{\uppercase{#1}}}}
\DeclareDocumentCommand\emds{ O{} }{\ensuremath{\displaystyle{#1}}}
\DeclareDocumentCommand\fillblanks{ O{default} }{\underline{\phantom{\huge #1#1}}}
\DeclareDocumentCommand\dfn{}{\textbf{\underline{Definition}: }}
\DeclareDocumentCommand\thm{}{\textbf{\underline{Theorem}: }}
\DeclareDocumentCommand\latex{}{\LaTeX\xspace}


%Numbers
\DeclareDocumentCommand\MN{ O{} }{\emds[\mathbb{N}^{#1}]}
\DeclareDocumentCommand\MZ{ O{} }{\emds[\mathbb{Z}^{#1}]}
\DeclareDocumentCommand\MQ{ O{} }{\emds[\mathbb{Q}^{#1}]}
\DeclareDocumentCommand\MR{ O{} }{\emds[\mathbb{R}^{#1}]}
\DeclareDocumentCommand\MC{ O{} }{\emds[\mathbb{C}^{#1}]}
\DeclareDocumentCommand\MF{ O{} }{\emds[\mathbb{F}^{#1}]}


%Calculus
\DeclareDocumentCommand\Mint{ O{f} O{x} O{} O{} }{\emds[\int_{#3}^{#4} #1 \,\, \mathrm{d} #2]}
\DeclareDocumentCommand\MD{ O{f} O{x} }{\emds[\frac{d #1}{d #2}]}

%Vector calculus
\DeclareDocumentCommand\ihat{}{\emds[\boldsymbol{\hat{\imath}}]}
\DeclareDocumentCommand\jhat{}{\emds[\boldsymbol{\hat{\jmath}}]}
\DeclareDocumentCommand\khat{}{\Muvec[k]}
\DeclareDocumentCommand\nhat{}{\Muvec[n]}
\DeclareDocumentCommand\tauhat{}{\Muvec[\tau]}
\DeclareDocumentCommand\Mpd{ O{f} O{x} }{\emds[\frac{\partial #1}{\partial #2}]}
\DeclareDocumentCommand\Mdint{ O{f} O{D} }{\emds[\iint_{#2} #1 \,\, \mathrm{d} A]}
\DeclareDocumentCommand\Miint{ O{f} O{x} O{} O{} O{y} O{} O{} }{\emds[\int_{#6}^{#7} \!\!\! \int_{#3}^{#4} #1 \,\, \mathrm{d} #2 \mathrm{d} #5]}
\DeclareDocumentCommand\Mtint{ O{f} O{E} }{\emds[\iiint_{#2} #1 \,\, \mathrm{d} V]}
\DeclareDocumentCommand\Msli{ O{f} O{C} }{\emds[\int_{#2} #1 \,\, \mathrm{d} s]}
\DeclareDocumentCommand\Mvli{ O{\Mvec{F}} O{C} }{\emds[\int_{#2} #1 \cdot \mathrm{d} \Mvec[r]]}
\DeclareDocumentCommand\Mvlib{ O{\Mvec{F}} O{C} }{\emds[\Msli[#1 \! \cdot \! \tauhat][#2]]}
\DeclareDocumentCommand\Mssi{ O{f} O{S} }{\emds[\iint_{#2} #1 \,\, \mathrm{d} S]}
\DeclareDocumentCommand\Mvsi{ O{\Mvec{F}} O{S} }{\emds[\iint_{#2} #1 \cdot \mathrm{d} \Mvec[S]]}
\DeclareDocumentCommand\Mvsib{ O{\Mvec{F}} O{S} }{\emds[\Mssi[#1 \! \cdot \! \nhat][#2]]}


%Linear Algebra
\DeclareDocumentCommand\Mvec{ O{x} }{\emds[\boldsymbol{\vec{#1}}]}
\DeclareDocumentCommand\Muvec{ O{u} }{\emds[\boldsymbol{\hat{#1}}]}
\DeclareDocumentCommand\vx{}{\Mvec[x]}
\DeclareDocumentCommand\vpi{}{\Mvec[\pi]}
\DeclareDocumentCommand\Mmat{ O{ccccccccccccccccccccc} O{1}}{\emds[\left[ \begin{array}{#1} #2 \end{array} \right]]}


%Analysis
\DeclareDocumentCommand\Mlim{ O{a_n} O{n \infty} }{\emds[\lim_{\small #2} #1]}
\DeclareDocumentCommand\Msup{ O{a_n} O{} }{\emds[\sup_{#2} #1]}
\DeclareDocumentCommand\Minf{ O{a_n} O{} }{\emds[\inf_{#2} #1]}
\DeclareDocumentCommand\Mlimsup{ O{a_n} O{n \to \infty} }{\emds[\limsup_{\small #2} #1]}
\DeclareDocumentCommand\Mliminf{ O{a_n} O{n \to \infty} }{\emds[\liminf_{\small #2} #1]}
\DeclareDocumentCommand\Mmax{ O{a_n} O{} }{\emds[\max_{#2} #1]}
\DeclareDocumentCommand\Mmin{ O{a_n} O{} }{\emds[\min_{#2} #1]}
\DeclareDocumentCommand\Mcup{ O{A_n} O{} O{} }{\emds[\bigcup_{#2}^{#3} #1]}
\DeclareDocumentCommand\Mcap{ O{A_n} O{} O{} }{\emds[\bigcap_{#2}^{#3} #1]}
\DeclareDocumentCommand\Msum{ O{a_n} O{} O{} }{\emds[\sum_{#2}^{#3} #1]}


%Comment severity
\DeclareDocumentCommand\sg{}{\textbf{(suggestion) }}
\DeclareDocumentCommand\mi{}{\textbf{(minor) }}
\DeclareDocumentCommand\mo{}{\textbf{(moderate) }}
\DeclareDocumentCommand\ma{}{\textbf{(major) }}
\makeatletter
\newcommand{\@coursenumber}{M5364-010}
\newcommand{\@coursetitle}{Data Mining 1}
\newcommand{\@semester}{Fall 2017}
\newcommand{\@school}{Tarleton State Univ}
\newcommand{\@instructor}{Dr. Scott Cook}


\newcommand{\@nameline}{\begin{flushright}Name: \rule{3in}{1pt}\end{flushright}}
\newcommand{\@studentName}{}
\newcommand{\studentName}[1]{\author{#1} \renewcommand{\@nameline}{}}

\newcommand{\@subtitle}{}
\newcommand{\subtitle}[1]{\renewcommand{\@subtitle}{\large{#1}\\}}

\newcommand{\@assignedDate}{}
\newcommand{\assignedDate}[1]{\renewcommand{\@assignedDate}{\hspace{2ex}Assigned #1}}

\newcommand{\@dueDate}{}
\newcommand{\dueDate}[1]{\renewcommand{\@dueDate}{\hspace{2ex}Due #1}}

\usepackage[T1]{fontenc}

%Sets page layout
\usepackage{geometry}
\geometry{verbose,tmargin=0.75in,bmargin=0.5in,lmargin=0.4in,rmargin=0.4in}

\usepackage{setspace}
\onehalfspacing
\setlength{\parindent}{1ex}
\setlength{\parskip}{1ex}

%Sets header and footer
\usepackage{fancyhdr}
\pagestyle{fancy}
\renewcommand{\headrulewidth}{0ex}
\renewcommand{\footrulewidth}{0ex}
\headsep = 0ex
\fancyhf{}
\usepackage[yyyymmdd]{datetime}
\settimeformat{ampmtime}
%Allows "page 2 of 7" in footer
\usepackage{lastpage}
\AtEndDocument{
	%\makescratch
	\label{LastPage}
}
\fancyhead[R]{\textbf{\large{\@author}}}
\fancyfoot[L]{Compiled \today \ at \currenttime}
\fancyfoot[R]{\thepage\ of\ \pageref{LastPage}}

%displays line numbers
\usepackage{listings}
\lstset{breaklines=true, breakatwhitespace=true, numbers=none, firstnumber=last}
\usepackage{lineno}
\renewcommand\linenumberfont{\footnotesize}


%Makes header of first page
\AtBeginDocument{
  \@nameline
	\begin{center}
	\textbf{
	\LARGE{\@title} \\
	\@subtitle
	\small{\@coursenumber--\@coursetitle \hspace{2ex} \@semester \hspace{2ex} \@school \hspace{2ex} \@instructor \@assignedDate \@dueDate}
	}
	\end{center}
	\vspace{0\parskip}
	\hhrule
    \vspace{\parskip}
} %Formatting, macros, package list, etc

\title{Syllabus}
%\subtitle{}
%\assignedDate{2016-08-30}
%\dueDate{2016-09-06}
\studentName{}
%\linenumbers

\usepackage{longtable}
\usepackage{tabularx}
\usepackage{import}
\begin{document}

\begin{center}
\vspace{-16\lineskip}
\begin{tabular}{ll||ll}
\textbf{Sec (010)}: & Math Bldg 304, TR 9:26-10:40AM & \textbf{My Office}: & Math Bldg 132 (254)968-1958 \\
\textbf{Instructor}: & Dr. Scott Cook & \textbf{Office Hours}: & TWR 1:00-3:00 or by appointment \\
\textbf{My Email}: & \href{mailto:scook@tarleton.edu}{scook@tarleton.edu} & \textbf{Math Clinic}: & Math Bldg 203 \\
\textbf{Math Dept Office}: & Math Bldg 142 (254)968-9168 & \textbf{Math Clinic Hours}: & MTWR 8:00-5:00, F 8:00-2:00\\
\end{tabular}
\hhrule
\end{center}


\bu[Materials]: 
\bit
  \item \underline{Required}: Anaconda distribution of Python, GitHub account, Amazon Web Services account
  \item \underline{Required}: \textit{A Whirlwind Tour of Python}, Jake Vanderplas, \href{https://github.com/jakevdp/WhirlwindTourOfPython}{\underline{github.com/jakevdp/WhirlwindTourOfPython}}
  \item \underline{Required}: \textit{Python Data Science Handbook}, Jake Vanderplas, \href{https://github.com/jakevdp/PythonDataScienceHandbook}{\underline{github.com/jakevdp/PythonDataScienceHandbook}} AND \href{http://shop.oreilly.com/product/0636920034919.do}{\underline{http://shop.oreilly.com/product/0636920034919.do}}
  \item \underline{Suggested}: \textit{The Elements of Statistical Learning}, Hastie, Tibshirani, Friedman, \href{https://web.stanford.edu/~hastie/ElemStatLearn/}{\underline{web.stanford.edu/~hastie/ElemStatLearn}}
  \item \underline{Suggested}: Many, many internet resources, newsletters, package documentation, etc.  This field is huge and changes fast.  You must learn to teach yourself using internet-based resources.  Hone this skill in this course.
\eit
 
\bu[Course Content]: Methods for data science including, but not limited to:
\bit
	\item Technology: python, numpy, pandas, scikit learn, matplotlib, seaborn, jupyter, keras (deep learning), regex
	\item Platforms: Amazon Web Services EC2, GitHub
	\item Supervised Learning Algorithms: Decision Trees, Naive Bayes, $k$-nearest neighbors, support vector machines, kernels, artificial neural networks \& deep learning, random forests/boosting/bagging/other ensemble methods
	\item Unsupervised Learning Algorithms: principle component analysis, k-means, gaussian mixtures, linear and logistic regression
	\item Model tuning \& evaluation: confusion matrices, accuracy/precision/recall/$F_1$, ROC curves, cost sensitive learning, cross-validation, hyper-parameter optimization, overfitting \& underfitting
\eit
	
\bu[Homework]: Homework will be assigned roughly once per week and due about 1 week later.  Assignments will be distributed and collected via GitHub, mainly in the form of Jupyter notebooks.  Each student will have one GitHub repository attached to the TSU organization where they will submit homework for grading.

\bu[Collaboration]: Collaboration is essential, especially for the coding work.  The mathematical ideas are complex; writing them into code that works on messy data is even more complex.  You will have bugs and errors of all sorts which will be difficult to fix without help.  Collaborate.  Then write up your own homework.  You must cite your collaborators to give them credit where it is due.  You will NOT lose points for crediting a classmate, but you might lose points if you don't when you should have.

\bu[Project]: You will do a capstone project using techniques from the course.  I will post instructions to GitHub.  You will give status reports in class periodically during the semester because your classmates often have good ideas for how to improve your project, avoid pitfalls, and solve technical challenges.

\bu[Exams]: There will be a comprehensive final exam Saturday, December 9, 8:00-10:30.

\bu[Contributions]: This field is huge, technically complex, and dynamic.  Our class needs to function as a team, trying to identify and present the best, most up to date resources available.  As graduate students, you need to do more than just complete the tasks I assign; you need to contribute to the class in ways that have not been specifically asked for.  There are many ways to do this, here are a few examples:
\bit
	\item In class, we struggle to understand some detail of an ML algorithm.  You find a great blog post that explains it clearly.  You post it on the GitHub repo and tell us a about it in class.
	\item You make really great suggestions for a classmate's project during their status report.
	\item You figure out a slick way to handle some IT difficulty.
\eit

\bu[Grading Policy]: The guaranteed grade cutoffs are listed below.  At my sole discretion, I may curve the course by relaxing them at the end of the semester.
\begin{center}
	\begin{tabular}{|c|c|} \hline
		Homework & 50\% \\ \hline
		Project & 30\% \\ \hline
		Final exam & 10\% \\ \hline
		Contributions & 10\% \\ \hline
	\end{tabular}
	\quad
	\begin{tabular}{|c|c|} \hline
		A & $[90\%,100\%]$\\ \hline
		B & $[80\%,90\%)$\\ \hline
		C & $[70\%,80\%)$\\ \hline
		D & $[60\%,70\%)$\\ \hline
		F & $[0\%,60\%)$\\ \hline
	\end{tabular}
\end{center}


%%%
%\bu[Very Rough Calendar (subject to change)]
%\begin{center}
%\begin{tabular}{|c|c|l|}%p{15cm}}%|p{1.3cm}|}
%\hline
%Week & Ch & Topics \\ \hline \hline
%1-3 & 1 & Logic \\
%3-5 & 2 & Sets, Graphs, Relations, Modular Arithmetic \\
%5-7 & 3 & Recursion \& Induction \\
%7-9 & 4 & Combinatorics \& Discrete Probability \\
%9-12 & 5 & Algorithms \& Complexity \\
%12-15 & 6 & Selected applications, catch up, course review \\
%\hline
%
%\end{tabular}
%\end{center}
%%%

%\bu[Calculator Policy]: TI-82 or better is recommended.  The TSU Math dept office has a limited number of calculators available to rent.

\bu[Notes]
\bit
  \item In the event that the university is closed for a scheduled class time, you should assume that whatever was scheduled/due on that day will be scheduled/due on the next class meeting.
  \item You are expected to present a valid TSU ID upon request.
  \item Aside from university and departmental policy, all aspects of course policy are at the discretion of the instructor and subject to change.
\eit

\hhrule

\begin{center}
\textbf{\LARGE{University Master Syllabus}}\\
\href{http://catalog.tarleton.edu/syllabus}{\underline{http://catalog.tarleton.edu/syllabus}}
\end{center}

\small
\bu[Catalog Description]: This course centers on the identification, exploration, and description of new patterns contained within data sets using appropriate software. Selected topics will be chosen from data exploration, classification, cluster analysis, and model evaluation and comparison.

\bu[Student Learning Objectives]: Upon completing this course, a student should be able to do the following: 
\ben[a)]
\item Examine raw data in order to detect data quality issues and interesting subsets or features contained within the data.
\item Transform raw data into a form appropriate for modeling.
\item Select and train appropriate models using the transformed data.
\item Measure the effectiveness of each model.
\item Draw appropriate conclusions.
\een



\bu[University Policy]: Students are responsible for knowing and abiding by the policies and information contained in the Tarleton Student Handbook [TSUSH].

\bu[Student Responsibilities]:  The student is solely responsible for:
\bit
	\item	Attending class.
	\item Completing every assignment by the specified due date.
	\item Utilizing, as needed, all available study-aid options (including meeting with the instructor, attending Supplemental Instruction (SI) sessions, going to the Math Clinic, using tutorial software, purchasing a student solutions manual, hiring a personal tutor, etc.) to resolve any questions that they might have regarding homework, course material, and/or technology projects.
	\item	Reading all relevant material in the course text and lecture.
	\item	Being present and prepared for each exam on the specified date and time, unless the instructor determines that a makeup exam is warranted (see Makeup Policy above).
  \item Obtaining assignments and other materials for classes from which they are absent.
  \item Giving as much effort as it takes to pass this course.
\eit

\bu[Student Success Statement - ADA]: It is the policy of Tarleton State University to comply with the \href{http://www.ada.gov}{\underline{Americans with Disabilities Act}} and other applicable laws. If you are a student with a disability seeking accommodations for this course, please contact the Center for Access and Academic Testing, at 254.968.9400 or \href{mailto:caat@tarleton.edu}{\underline{caat@tarleton.edu}}. The office is located in Math 201. More information can be found at \href{http://www.tarleton.edu/caat}{\underline{www.tarleton.edu/caat}} or in the University Catalog.


\bu[Cell phones]: Students are expected to set their cell phone so as to emit no audible noise in the classroom. Except for emergency situations, cell phone use (including texting) during the class period is prohibited. A student who is noticeably (to the instructor) distracted by his/her cell phone and/or distracting others with it may be asked to immediately disable it or to leave the classroom.  To compensate for your electronic deprivation, keep your calculator on.

\bu[Absence Policy]:  Class absence policies will be established and enforced by each individual course instructor.  The course instructor may recommend to the Dean of Students that a student be dropped from a course if excessive absences prevent satisfactory progress.

\bu[Makeup Policy]:  Each course instructor has the responsibility and authority to determine if work can be made-up because of absences.  Students may request make-up considerations for valid and verifiable reasons such as the following:
\bit
\item Illness
\item Death in the immediate family
\item Legal proceedings
\item Participation in sponsored University activities (It is the responsibility of students who participate in University-sponsored activities to obtain a written explanation for their absence from the faculty/staff member who is responsible for the activity.)
\eit

%\bu[Drop Policy]:  A student who withdraws from a course before the thirteenth class day of a regular semester or before the fifth class day in a summer term receives no grade, and the course will not be listed on that student's permanent record.  A student who withdraws from a course before the end of the tenth week of a regular semester or the fourteenth class day of a summer term receives a grade of W.

\bu[Failing grades] Tarleton differentiates between a failed grade in a class because a student never attended (F0 grade), stopped attending at some point in the semester (FX grade), or because the student did not pass the course (F) but attended the entire semester. These grades will be noted on the official transcript. Stopping or never attending class is considered an unofficial withdrawal and can result in the student having to return aid monies received.  For more information see the Tarleton Financial Aid website.


%\bu[Day of Service - April 7th] - In support of Tarleton's core value of service, each student is expected to participate in a service learning experience as a part of the Spring term week of service.  This experience will challenge students to be engaged in the local community, address a community need, connect course objectives to the world around you, and involve structured student reflection. In this service learning experience you will not only enhance your knowledge and skills, but actively use those skills as you serve your community.


\bu[Student Safety and Title IX]: You are in college to achieve academic success, but you must feel safe and take care of yourself to reach your full potential. You have the right to pursue your education in a safe environment. Title IX makes it clear that violence and harassment based on sex and gender are civil rights offenses subject to accountability. \textit{If you or someone you know has been harassed or assaulted, there is help and support on campus}. You may seek assistance confidentially through the Student Counseling Center or the Student Health Center. You may also make a report to the campus Title IX coordinator, which may trigger a university investigation (not a criminal investigation). Additionally, you may pursue criminal charges through the university police department. If the assault occurred away from campus, UPD can assist you in connecting with the appropriate law enforcement agency.
\begin{center}
Student Counseling Center: 254-968-9044 (phone is answered 24 hours a day, 7 days a week), TSC 212

Student Health Services: 254-968-9271, TSC 212 

Title IX Coordinator: 254-968-9754, Admin Annex 1, Room 112

University Police Department: 254-968-9002, located on the back side of Wisdom Gym
\end{center}

\href{http://www.tarleton.edu/strategicplan/2016-2020/mission-vision.html}{\underline{University Core Values}}

\bu[Academic Integrity Statement]:  Tarleton State University's core values are integrity, leadership, tradition, civility, excellence, and service.  Central to these values is integrity, which is maintaining a high standard of personal and scholarly conduct.  Academic integrity represents the choice to uphold ethical responsibility for one's learning within the academic community, regardless of audience or situation.

\bu[Academic Civility Statement]:  Students are expected to interact with professors and peers in a respectful manner that enhances the learning environment. Professors may require a student who deviates from this expectation to leave the face-to-face (or virtual) classroom learning environment for that particular class session (and potentially subsequent class sessions) for a specific amount of time. In addition, the professor might consider the university disciplinary process (for Academic Affairs/Student Life) for egregious or continued disruptive behavior.

\bu[Academic Excellence Statement]:  Tarleton holds high expectations for students to assume responsibility for their own individual learning.  Students are also expected to achieve academic excellence by:
\bit
  \item honoring Tarleton's core values. 
  \item upholding high standards of habit and behavior.
  \item maintaining excellence through class attendance and punctuality.
  \item preparing for active participation in all learning experiences. 
  \item putting forth their best individual effort.
  \item continually improving as independent learners.
  \item engaging in extracurricular opportunities that encourage personal and academic growth.
  \item reflecting critically upon feedback and applying these lessons to meet future challenges.
\eit

\bu[Academic Honesty]:  Tarleton State University expects its students to maintain high standards of personal and scholarly conduct.  Students guilty of academic dishonesty are subject to disciplinary action.  Academic dishonesty includes, but is not limited to, cheating on an examination or other academic work, plagiarism, collusion, and the abuse of resource materials.  The faculty member is responsible for initiating action for each case of academic dishonesty that occurs in his or her class.

\bu[Academic dishonesty] includes, but is not limited to, cheating on an examination or other academic work, plagiarism, collusion, unauthorized use of technology and the abuse of resource materials.
\ben
  \item Academic work means the preparation of an essay, thesis, problem, assignment or other projects submitted or completed for course credit and to meet other requirements for noncourse credit.
  \item What constitutes an act of academic dishonesty may, in part, depend on the particular course and expectations of academic integrity in the context of the course objectives. This includes, but is not limited to, the following:
  \ben
    \item Copying, without instructor authorization, from another student's test paper, laboratory report, other report, computer fi les, data listing and/or programs.
    \item Using, during a test, materials not authorized by the person giving the test. 
    \item Collaborating with another person without instructor authorization during an examination or in preparing academic work.
    \item Knowingly and without instructor authorization, using, buying, selling, stealing, transporting, soliciting, copying, or possessing, in whole or in part, the contents of an unadministered test or other required assignment.
    \item Substituting for another student or permitting another person to substitute for oneself in taking an examination, preparing academic work, or attending class.
    \item Bribing another person to obtain an unadministered test or information about an unadministered test.
    \item Using technological equipment such as calculators, computers or other electronic aids in taking of tests or preparing academic work in ways not authorized by the instructor or the university.
  \een
  \item Plagiarism means the appropriation of another's work and the unacknowledged incorporation of that work in one's own written work in any academic setting. 
  \item Collusion means the unauthorized collaboration with another person in preparing written work in any academic setting.
  \item Abuse of resource materials means the mutilation, destruction, concealment, theft or alteration of materials provided.
\een

\begin{center}
	\textbf{This syllabus subject to change as deemed appropriate by the instructor.}
\end{center}

\end{document}